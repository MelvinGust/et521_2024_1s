% Relatorio 01 : Propriedades de Circuitos Magnéticos
% ---------------------------------------------------
% 
% Este documento está formatado seguindo o que cheguei a entender das normas ABNT com algumas mudanças para facilitar nossa vida no momento de formatar ele. Caso seja desnecessário ou haja um erro, sintam-se a vontade de modificar o formato do documento.
% Note-se que o formato que estou utilizando foi montado em cima de um que o Lucas Oliver tinha feito, pelo qual pode ser que hajam pacotes ou algums "detalhes" desnecesários. Se for o caso, podemos trocar sem problema.
%
% Dúvidas para tirar: 
% - Quanto seria um interlinhado de 1,5? Esse tamanho tanto em LaTeX como em Word não possui convenção, fazendo que ele seja distinto dependendo de como o implemente (com setspace ou sem). Exemplo: https://tex.stackexchange.com/questions/65849/confusion-onehalfspacing-vs-spacing-vs-word-vs-the-world
% - Iremos usar Biblatex? Se sim, talvez hajam bugs no momento de compilar. Se não, iremos redatar a bibliografia de forma manual.

% Pacote usado para definir formatação básica do pdf, como tamanho de fonte e página.
\documentclass[12pt, a4paper, notitlepage]{article}

% Pacotes usados para escrever em portugûes brasileiro.
\usepackage[T1]{fontenc}
\usepackage{lmodern} % Sua inclusão é importante para ter boa qualidade no PDF.
\usepackage[utf8]{inputenc}
\usepackage[english, brazil]{babel}
\usepackage{bookmark}
\usepackage[hypcap=true, font=scriptsize, center]{caption}

% Pacotes utilizados para definir a geometria e tamanho das páginas
\usepackage{geometry} 
\geometry{
    a4paper,
    left=30mm, right=20mm,
    top=30mm, bottom=20mm
}
\usepackage{setspace} % Note-se que o uso de setspace é chato. Dependendo do jornal, é uma boa ideia mudar.
\onehalfspacing
% \usepackage{indentfirst} % Este pacote só serve para ter interlinhado no começo de cada parragrafo. Na minha visão pessoal, o documento fica feio assim, mas existem as convenções por algum motivo.

% Configuração de Biblatex para citações
\usepackage[style=numeric, backend=biber]{biblatex} % Isso aqui é só para botar referências bonitas.
\addbibresource{refs.bib}
\nocite{*}

\usepackage{enumitem}
\setenumerate{noitemsep}

\usepackage{chngcntr, amsmath, amsthm, amsfonts, fancyhdr, float, graphicx, hyperref, xcolor, multicol, multirow, pgfplots, titlesec, titling, soul, subcaption}

\usepackage{esint} % Este pacote permite a escrita de integrais fechadas no documento.

\newcounter{counterquestions}

\newenvironment{questions}{
    \noindent
    \stepcounter{counterquestions}
    \textbf{Questão\:\thecounterquestions\:-}
    \noindent
}{
    \noindent
}

% \newlist{questions}{enumerate}{1}
% \setlist[questions]{label=\arabic*., wide=0pt, font=\bfseries}
% \let\question=\item

\graphicspath{ {../Imagens/Sem01} } % Caminho para acessar as imagens que iremos utilizar neste documento.

% \pgfplotsset{width=7.5cm, compat=1.18, every tick label/.append style={font=\tiny}}
\tikzset{>=latex}

% Aqui temos um mente de configurações que talvez sejam interessantes para a gente.
\pagestyle{fancy}
\fancyhf{}
\lhead{Relatório 01}
\chead{}
\rhead{2024.1}
\lfoot{}
\cfoot{}
\rfoot{\thepage}
\setlength{\parindent}{0pt}
\setlength{\parskip}{0.2cm}
%\setlength{\parindent}{.5cm} % Valor padrão. Norma ABNT diz que deve ser 1.5cm.
\setlength{\headheight}{15pt} % Requisito para compilar
\setlength{\parsep}{0pt}
\setlength{\topsep}{0pt}
\setlength{\partopsep}{0pt}
\setlength{\belowcaptionskip}{0pt}
\setlength{\itemsep}{1em}

% Esta seção formata os dois tipos de títulos que acredito podemos utilizar: Section e Subsection.
\titleformat{\section}{\bfseries\Large}{}{0pt}{} % Section pra dividir a introdução, experimento e bibliografia.
\titleformat{\subsection}{\bfseries\large}{}{0pt}{(\roman{subsection}) } % Subsection pra cada parte do experimento.


% Criei isso aqui pra uniformizar os itens dos experimentos.
\newenvironment{expitem}{\par\medskip\noindent\minipage{\linewidth}\setlength{\parindent}{1.5em}\bfseries\noindent\textbullet\ }{\endminipage\par\smallskip}

\begin{document}        
    \def\figscale{0.8} 
     % Multiplicador do tamanho da imagem em relação ao comprimento da linha. Atualmente em 80%.
    
    % Os contadores das equações, figuras e tabelas agora são globais e não dependem mais de outros contadores
    \counterwithout{equation}{section}
    \counterwithout{figure}{section}
    \counterwithout{table}{section}
    
    \begin{titlepage}
        \centering
    
        {\LARGE \textsc{Universidade Estadual de Campinas}\par}
        {\Large \textsc{Faculdade de Engenharia Elétrica e de Computação}\par}
        \vspace{1cm}
        {\Large \textsc{Relatório 01}\par}
        \vspace{1.5cm}
        {\huge\bfseries Propriedades de Circuitos Magnéticos\par}
        \vspace{2cm}
        \textbf{Alunos}\par
        Bianca Giovanna de Castro Fernandez (166973) \par\par
        Ivan de Sousa Oliveira (206473) \par\par
        Melvin Gustavo Maradiaga Elvir (185068) \par\par
        Vinicius dos Santos Ribeiro (206643) \par
        \vfill
    % Bottom of the page
        {\large \today\par}
    \end{titlepage}
    
    \newpage
        \noindent
        \section{Introdução}

        \section{Discussão}
        \begin{questions}
            Como é possível obter tensões (ou correntes) elétricas a partir de fluxo magnético? Exemplifique com duas alternativas distintas.
        \end{questions}
            
        A partir do fluxo magnético, é possível obter tensão e corrente elétrica por conta da Lei de Faraday e por conta do comportamento encapsulado na forma integral da equação de Maxwell-Faraday \cite{maxwell1}:
        \begin{align*}
            \oint_{C}\vec{E}\circ d\vec{l} &=-\int_{S}\frac{\partial \vec{B}}{\partial t}\circ\hat{n}dA \\
            fem = -\frac{d}{dt}\int_{S}\vec{B}\circ\hat{n}dA 
        \end{align*}
        Observe-se que segundo a Lei de Faraday, ao induzirmos um fluxo variante no tempo, iremos gerar uma força eletromotriz (fem) em \textbf{oposição} a essa mudança de fluxo.
        Olhando agora a equação de Maxwell-Faraday, perceba-se que toda mudança na densidade de fluxo magnético passando no meio de uma superficie
        
        
        \begin{questions}
            Na distribuição do campo magnético obtida através de um eletroímã com núcleo em forma de U sob a placa de acrílico:
            \begin{itemize}[noitemsep,topsep=0pt]
                \item[\textbf{(a)}]{Existe influência da natureza da corrente (contínua ou alternada)? Por quê?}
                \item[\textbf{(b)}]{A presença de material metálico sob a placa resulta em alguma modificação na distribuição do campo magnético? Exemplifique.}
            \end{itemize}
        \end{questions}
        
        \begin{questions}
            Na montagem identificada como anéis de Thompson:
            \begin{itemize}[noitemsep,topsep=0pt]
                \item[\textbf{(a)}]{Existe influência da natureza da corrente (contínua ou alternada)? Por quê?}
                \item[\textbf{(b)}]{A presença de material metálico sob a placa resulta em alguma modificação na distribuição do campo magnético? Exemplifique.}
            \end{itemize}
        \end{questions}
        
        \begin{questions}
            Desenhe o circuito utilizado para a visualização do laço de histerese do material magnético de um dispositivo eletromagnético e esboce os laços de histerese obtidos com a bobina em núcleo fechado e com um dos enrolamentos do transformador preto. Comente a respeito.
        \end{questions}
        
        \begin{questions}
            O que é essencial para o funcionamento tanto do experimento de Ruhmkorff como da ignição automotiva tradicional? Qual é a diferença básica entre eles, sob o ponto de vista eletromagnético
        \end{questions}
        \newpage
        
        https://www.papeeria.com/join?token_id=371b920d-6bb6-4722-a118-f79ef5f37311&retry=3
        
        https://www.papeeria.com/join?token_id=371b920d-6bb6-4722-a118-f79ef5f37311&retry=3
        
        https://www.papeeria.com/join?token_id=371b920d-6bb6-4722-a118-f79ef5f37311&retry=3
        
        https://www.papeeria.com/join?token_id=371b920d-6bb6-4722-a118-f79ef5f37311&retry=3
     % Sem isso LaTeX precisa de citações para acrescebtar a bibliografia
    % \printbibliography

\end{document}
